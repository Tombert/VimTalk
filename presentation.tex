\documentclass{beamer}
\usepackage{default}
% colors (theme=...): red (default), blue, cyan, orange, green
\mode<presentation>

% title and author
\title{Vim Talk}
\author{Tom Gebert}

% document
\begin{document}

\begin{frame}{Some Obligatory History}
  \begin{itemize}
	  \item Bill Joy created vi in 1976 for Unix \begin{itemize}
			  \item Continuation of the ``ed'' line editor
	  \end{itemize}
          \item Vim stands for ``Vi IMproved'' and was released in 1991
	  \item Started as a basic port but has added a bunch of new features over vanilla vi \begin{itemize}
			\item Default distribution with most flavors of Linux
	  \end{itemize}
		  
          
  \end{itemize}
\end{frame}

\begin{frame}{Modes!}
	The part that scares everyone away from Vim is the concept of ``modal editing''. What does this mean? 
	\begin{itemize}
		\item You can be in one of three different modes at any point \begin{itemize}
				\item \texttt{i} or \texttt{a} puts you in insert mode
				\item \texttt{v} puts you in visual (or ``highlight'') mode
				\item Exit either of those modes by hitting escape.  This puts you in Normal mode (which is also the default)
		\end{itemize}
		
		\item It usually says on the bottom which one you're in. 
		\item If you're ever scared, just hit escape.  It will be OK. 
	\end{itemize}
\end{frame} 
\begin{frame}{Simple Movement}
  There is a relatively non-intuitive set of keys for moving around character-by-character
  \begin{itemize}
	  \item \texttt{h} moves left
	  \item \texttt{l} moves right 
	  \item \texttt{j} moves down 
	  \item \texttt{k} moves up 

          
  \end{itemize}
  \begin{itemize}
	  \item You can also use the arrow keys
          
  \end{itemize}
\end{frame}
\begin{frame}{More advanced movement}
	I almost never use the \texttt{hjkl} commands to move around, and prefer to usually go by word or by paragraph.  
	\begin{itemize}
		\item \texttt{w} moves you forward a word and puts your cursor in front of the next word. 
		\item \texttt{e} moves you to the end of the current word or the end of the next one
		\item \texttt{\}} moves you to the end of the paragraph (or the next one)
		\item \texttt{\{} moves you to the beginning of the paragraph (or the previous one)
	\end{itemize}
	I find it's useful to get into the pattern of going ``by word'' instead of ``by character'' as it makes writing macros a lot simpler
\end{frame}
\begin{frame}{More advanced movement}
	\begin{itemize}
		\item \texttt{0} jumps to the beginning of the line
		\item \texttt{\$} jumps to the end of the line
		\item \texttt{I} jumps to the beginning of the line and puts you in insert mode
		\item \texttt{A} jumps to the end of the line and puts you in insert mode
	\end{itemize}
\end{frame}
\end{document}
