\documentclass{beamer}
\usepackage{default}
\usepackage{graphicx}
% colors (theme=...): red (default), blue, cyan, orange, green
\mode<presentation>
\usecolortheme{wolverine}
% title and author
\title{Getting Started with Vim}
\author{Tom Gebert}

% document
\begin{document}
  \frame{\titlepage}
\begin{frame} {What is Vim}
	What is Vim? 
	\begin{itemize}
		\item Vim is a declarative text editor by Bram Moolenaar
		\item Vim is an editing ideology. 
		\item Vim is life. 
	\end{itemize}
	What's the point of learning it? 
	\begin{itemize}
		\item Vim has been ported to effectively every system imaginable. 
		\item Vim (or vi) is pre-installed on most *nix systems \begin{itemize} 
		   \item Particularly useful if you SSH
		\end{itemize}
		\item The keystrokes and ideology behind it can be incredibly powerful when used right
	\end{itemize}

\end{frame}
\begin{frame} {Installation}
	On Mac: \begin{itemize}
		\item \texttt{brew install vim} or \texttt{brew install macvim} (for GUI)
		\item The built-in version of Vim for Mac is pretty out-of-date but should still work
                \end{itemize}
	For Windows: \begin{itemize}
		\item Download and install GVim
		\item OR download the VsVim plugin for Visual Studio (what I'll be using for this demonstation)
	\end{itemize}

	All the demos for this talk should work on any of the above.  
\end{frame}
\begin{frame}{Some Obligatory History}
  \begin{itemize}
	  \item Bill Joy created vi in 1976 for Unix \begin{itemize}
			  \item Continuation of the ``ed'' line editor
	  \end{itemize}
          \item Vim stands for ``Vi IMproved'' and was released in 1991
	  \item Started as a basic port but has added a bunch of new features over vanilla vi \begin{itemize}
			\item Default distribution with most flavors of Linux
	  \end{itemize}
		  
          
  \end{itemize}
\end{frame}

\begin{frame}{Modes!}
	The part that scares everyone away from Vim is the concept of ``modal editing''. What does this mean? 
	\begin{itemize}
		\item You can be in one of three different modes at any point \begin{itemize}
				\item \texttt{i} or \texttt{a} puts you in insert mode
				\item \texttt{v} puts you in visual (or ``highlight'') mode
				\item Exit either of those modes by hitting escape.  This puts you in Normal mode (which is also the default)
		\end{itemize}
		
		\item It usually says on the bottom which one you're in. 
		\item Makes learning a bit tricky, but also limits the need for modifier keys
		\item If you're ever scared, just hit escape.  It will be OK. 
	\end{itemize}
\end{frame} 
\begin{frame}{Simple Movement}
  There is a relatively non-intuitive set of keys for moving around character-by-character
  \begin{itemize}
	  \item \texttt{h} moves left
	  \item \texttt{l} moves right 
	  \item \texttt{j} moves down 
	  \item \texttt{k} moves up 

          
  \end{itemize}
  \begin{itemize}
	  \item You can also use the arrow keys
          
  \end{itemize}
\end{frame}
\begin{frame}{More advanced movement}
	I almost never use the \texttt{hjkl} commands to move around, and prefer to usually go by word or by paragraph.  
	\begin{itemize}
		\item \texttt{w} moves you forward a word and puts your cursor in front of the next word. 
		\item \texttt{e} moves you to the end of the current word or the end of the next one
		\item \texttt{b} jumps to the beginning of the current word (or the next one)
		\item \texttt{\}} moves you to the end of the paragraph (or the next one)
		\item \texttt{\{} moves you to the beginning of the paragraph (or the previous one)
	\end{itemize}
	I find it's useful to get into the pattern of going ``by word'' instead of ``by character'' as it makes writing macros a lot simpler
\end{frame}
\begin{frame}{More advanced movement}
	You can repeat a movement by putting a number in front. For example, \texttt{20w} will jump forward 20 words
\end{frame}
\begin{frame}{More advanced movement}
	\begin{itemize}
		\item \texttt{0} jumps to the beginning of the line
		\item \texttt{\$} jumps to the end of the line
		\item \texttt{I} jumps to the beginning of the line and puts you in insert mode
		\item \texttt{A} jumps to the end of the line and puts you in insert mode
		\item \texttt{gg} jumps to the beginning of the file
		\item \texttt{G} jumps to the end of the file
	\end{itemize}
\end{frame}
\begin{frame}{Actions}
	Movement is great, but the bread and butter of Vim is how it can compose with actions
	\begin{itemize}
		\item \texttt{d} begins the delete action
		\item \texttt{y} begins the yank (copy) action
		\item \texttt{v} begins the visual (highlight) action. \begin{itemize}
				\item Visual can be an action \textbf{and} a mode
		\end{itemize}
	\end{itemize}
	Any of these actions can be combined with the movements from before. 

		\textbf{Note}: Once an action has been started, \texttt{i} basically means ``current''

\end{frame}
\begin{frame}{Examples}
	As a demonstration, try these in Normal Mode
	\begin{itemize}	
		\item Delete the next twenty words: \texttt{d20w}
		\item Delete the next two characters: \texttt{d2l}
		\item Copy the previous eight words: \texttt{y8b}
		\item Select until the end of the next four paragraphs: \texttt{v4\}}
		\item Delete everything after this point on this line: \texttt{d\$}
	\end{itemize}
\end{frame}
\begin{frame}
	\begin{itemize}	
		\item Copy everything before this point on this line: \texttt{y0}
		\item Select everything from this point to the end of the file \texttt{vgg}
		\item Delete the current word: \texttt{diw}
		\item Copy everything in the paratheses: \texttt{yib} \begin {itemize}
		\item Presumably to be annoying, after an action, \texttt{b} means "small block" (parentheses) in this context. \texttt{B} means ``big block'' and works with curly-braces. 
	        \end{itemize}
	\end{itemize}
	\end{frame}
\begin{frame}{Shortcuts}
  Some of these are so common that they have easier shortcuts
  \begin{itemize}
	  \item \texttt{x} is basically equivalent to \texttt{dl} and deletes the current character
	  \item \texttt{yy} is basically equivalent to \texttt{0y\$} and copies the current line
	  \item \texttt{dd} is basically equivalent to \texttt{0d\$} and deletes the current line
  \end{itemize}
\end{frame}
\begin{frame}{Macros}
   Keeping your editing style declarative lends itself incredibly well for the use of macros
   \begin{itemize}
	   \item begin recording a macro by hitting \texttt{q} and any other letter (for this we'll use \texttt{a})
           \item Do a series of commands 
	   \item End the macro by hitting \texttt{q} again
	   \item Execute the macro by hitting \texttt{@a} (substitute the \texttt{a} with whatever letter you chose)
	   \item Repeat the macro by hitting \texttt{@@}
   \end{itemize}
\end{frame}
\begin{frame}{Example: Converting Hyperlinks to JSON}
	Here's an example 
\end{frame}

\begin{frame}{Example: Surround with delimiters and capitalize}
	Here's another example.
\end{frame}
\begin{frame} {Basic Macro Tips}
	Here are a few basic things I do when creating macros. 
	\begin{itemize}
		\item Start by hitting 0 so that you can start at the beginning of the line
		\item Think as declaratively as possible.  \begin{itemize}
				\item Move by words instead of by character so that the macro is easier to reuse
				\item Use the "jump to" features like "jump to end" liberally
				\item Before writing the macro, look for bits that are easy to generalize and try and ignore the bits that aren't
				\item When you're done making your edits, hit \texttt{j} to move to the next line, so that the macro is easier to repeat
		\end{itemize}
			\textbf{Note}: None of these are law, just rules of thumb for getting started. 
	\end{itemize}
\end{frame}
\begin{frame} {To Learn More}
	\begin{itemize}
		\item On Mac and Linux, type \texttt{vimtutor} into the command line. \begin{itemize}
				\item On Windows, it comes with either Git Bash or Cygwin
	        \end{itemize}
	\end{itemize}
\end{frame}
\begin{frame} {To Learn More}
	\begin{itemize}

	\item Avoid the temptation to live in Insert Mode \begin{itemize}
			\item It's best to \textbf{only} use insert mode when you're typing new text. 
			\item Inserts count as only \textbf{one} layer of undo-history. 
			\item Editor is wholly unremarkable in insert mode
	\end{itemize}
	\end{itemize}
\end{frame}
\begin{frame}{To Learn More}
	Experiment!
	\begin{itemize}
		\item Vim's interface is designed to be composable and guessable.  
		\item It's entirely possible that you can discover a new and useful command that isn't explicitly documented
		\item It makes the editor a lot more fun to use
	\end{itemize}
\end{frame}

\begin{frame}{To Learn More}
	Feel free to bother me for questions!  
	\begin{itemize}
		\item @tombert on Slack (the guy with the smiley-icon \includegraphics[scale=.006]{awesome.jpg})
		\item github.com/tombert
		\item tom.gebert@jet.com
	\end{itemize}
\end{frame}
\end{document}
